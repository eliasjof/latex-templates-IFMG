\chapter{Revisão Bibliográfica} \label{RevisaoBibliografica}

Neste capítulo, que também pode ser chamado de Referencial Teórico, deve ser feita uma revisão bibliográfica apresentando um resumo com as discussões já feitas por outros autores sobre o assunto abordado.

Para fazer a revisão bibliográfica é necessário consultar os trabalhos  realizados por outros autores sobre a temática escolhida para ser desenvolvida. Devem ser apresentados os conceitos mais importantes, justificativas e características sobre o assunto abordado, do ponto de vista da analise feita pelos autores. 

Descreva os resultados já alcançados, indicando os respectivos responsáveis, e finalize o capítulo apresentando as diferenças entre os trabalhos citados e o que será desenvolvido, destacando a sua contribuição.

Este capítulo torna-se interessante quando é preciso fornecer uma fundamentação teórica e/ou explicações prévias para o leitor (considerando que este seja leigo no assunto) antes de introduzi-lo ao capítulo da metodologia desenvolvida.

Para citar utilize o comando $\backslash cite\{nomedereferencia\} $. O arquivo bibliografia.bib deve ser preenchido corretamente. Veja o exemplo de citação da norma \cite{NBR10520}, de um livro \cite{ogata} e de um artigo \cite{sbse} e o exemplo de citação de um autor \citeonline{Elfes1989grid}.

Ou citação direta:

\begin{flushright}
\noindent
\parbox{\linewidth-4cm}{
“As Universidades e Faculdades que há alguns anos atuavam de forma passiva nas questões educacionais, principalmente nas relações com o mercado, hoje estão sendo forçadas a ser proativas em suas ações estratégicas, principalmente na identificação e satisfação das expectativas e necessidades de um mercado cada vez mais seletivo e exigente. É fundamental formar cidadãos capazes de atuar na sociedade, de conhecer seus direitos e deveres, de compreender o que se passa no mundo.”  \cite{neves2002}.
}
\end{flushright}


Exemplo de como utilizar itens no Latex:

\begin{itemize}\itemsep5pt
    \item item 1
    \item item 2
    \item item 3
\end{itemize}



\section{Acrescentando um arquivo tex na estrutura}

Como acrescentar uma nova subseção, utilizando um arquivo externo:

\begin{enumerate}
    \item crie um arquivo .tex (ex.: meuarquivo.tex)
    \item Se for um arquivo de capítulo:
        \subitem No arquivo Monografia.text acrescente a seguinte linha na ordem que deseja aparecer no texto: $\backslash include\{Capitulos/meuarquivo\}$
    \item Se for parte do texto:
        \subitem Inserir no arquivo onde se deseja continuar o seguinte comando:
            \subsubitem $\backslash input\{Capitulos/meuarquivo\}$
    
    
\end{enumerate}


{\color{blue}

Obs.: Este texto foi escrito no arquivo exemplo subsecao2.tex. 
Note que o arquivo é inserido em continuidade na página!

}
 % a ordem é a que aparece no texto





